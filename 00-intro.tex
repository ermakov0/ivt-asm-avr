\documentclass[main.tex]{subfiles}

\begin{document}
\chapter*{Введение}
\addcontentsline{toc}{chapter}{Введение}

Дисциплина <<Микропроцессорные системы>> входит в учебные планы подготовки бакалавров направления 09.03.01 <<Информатика и вычислительная техника>>. В качестве самостоятельной работы студенты заочной формы обучения выполняют контрольную работу.

Основная нагрузка по освоению студентом программного материала ложится на самостоятельную работу. Первостепенное значение при этом придаётся формированию навыков и умений решения учебных проблем и познавательных задач, а именно:

\begin{itemize}
\item анализу получаемой и добываемой информации;
\item сопоставлению и разбору различных точек зрения;
\item выдвижению исследовательских гипотез и их доказательству;
\item ценностной ориентации в незнакомом тексте;
\item анализу отдельного факта или группы фактов;
\item изложению собственного мнения.
\end{itemize}

Все эти задачи студент реализует при написании контрольной работы, то есть от студента требуется осуществление практических действий по схеме <<информация~--- знания~--- деятельность~--- интернет~--- новое знание>>. За счёт этого происходит переход от простого накопления знаний к уровню их применения.

При выполнении контрольной работы необходимо использовать не менее трёх первичных источников.


\end{document}
